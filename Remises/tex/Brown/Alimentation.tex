Pour alimenter ad�quatement tous les syst�mes pr�sents sur le robot, on doit avoir une batterie avec un voltage qui se situe entre 21V et 30V pour alimenter le r�gulateur de l'ordinateur embarqu�, qui fonctionne � 19V avec un courant de 3.5A. On doit �galement avoir assez de puissance pour que les moteurs et l'�lectronique de contr�les puissent fonctionner pendant au moins dix minutes. La puissance dont le robot a besoin se d�finit surtout par celle des moteurs et des servomoteurs. Les moteurs demandent au maximum 800mA � 12V et les servomoteurs qui servent � contr�ler la cam�ra demandent environ 50mA � 5V. On utilise �galement un servomoteur plus puissant pour d�placer le pr�henseur, qui demande environ 1A � 5V. En additionnant la puissance de ces syst�mes, la puissance requise est de 110W. On a choisi une batterie LiPO 6S de 4 500mA, ce qui peut donner 27A, pour un total de 599W en 10 minutes. Les six cellules sont n�cessaires pour avoir 22.2V, ce qui permet de donner 19V pour l'ordinateur embarqu�. Le 4 500mA est justifi�, car on veut pouvoir travailler sur le robot plus longtemps que 10 minutes, pour pouvoir faire des tests. Nous avons calcul� que cette batterie pourrait nous donner environ 50 minutes d'autonomie. 

L'utilisation d'une batterie LiPO requi�re un chargeur intelligent qui refuse de charger la batterie si les cellules ont un voltage trop faible, car la batterie pourrait exploser. On a donc achet� un tel chargeur et on utilise �galement un sac de protection pour diminuer l'impact d'une �ventuelle explosion. L'utilisation d'une LiPO requi�re �galement un syst�me qui donne une alarme lorsqu'il faut recharger la batterie, pour justement emp�cher les risques d'explosions. Cette fonction est assur�e par un petit afficheur de voltage, con�u sp�cialement pour les batteries LiPO, qui sonne quand la tension des cellules est trop faible. 

Le robot a donc besoin de trois niveaux de tension diff�rents pour fonctionner. On utilise des r�gulateurs pour avoir une tension stable, aux voltages d�sir�s. Le r�gulateur de l'ordinateur embarqu� est fourni, alors il reste les r�gulateurs pour le 12V et le 5V � trouver. On a choisi de prendre deux fois le m�me r�gulateur, qui prend 4V � 38V en entr�e et 1.25V � 36V en sortie. Ce r�gulateur peut fournir 5A et il a un afficheur de voltage pour conna�tre facilement le niveau de tension en sortie. 