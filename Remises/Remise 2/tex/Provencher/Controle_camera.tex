Le contr�le de la cam�ra embarqu�e s'effectue par des servomoteurs contr�l�s par le Micro Maestro 6-Channel USB Servo Controller. Afin de simplifier les commandes n�cessaires au contr�le de la position de la cam�ra, quatre positions pr�d�termin�es ont �t� programm�es : gauche, droite, avant, tr�sor. Comme ces quatre positions permettent de couvrir enti�rement le champ de vision int�ressant, il n'est pas n�cessaire d'ajouter davantage de possibilit�s de contr�le de cam�ra. La position tr�sor correspond � la position avant dans laquelle la cam�ra regarde compl�tement vers le bas. Afin de d�terminer ces positions, le logiciel Maestro Control Center a �t� utilis� pour noter les valeurs nominales de ces positions. Une fois les valeurs de position en main, une librairie Arduino a �t� d�velopp�e afin de d�placer la position de la cam�ra dans ces diff�rentes valeurs.
\medbreak
Pour communiquer entre le Arduino Mega 2560 et le contr�leur de servomoteur, une communication UART est �tablie afin de pouvoir envoyer les commandes de positions au contr�leur de servomoteur. Comme il n'est pas n�cessaire de recevoir de l'information de la part du contr�leur de servomoteur, uniquement la branche TX du Arduino Mega est utilis�e. Le contr�leur de servomoteur est �galement reli� au module d'alimentation afin de fournir le courant n�cessaire aux servomoteurs.
