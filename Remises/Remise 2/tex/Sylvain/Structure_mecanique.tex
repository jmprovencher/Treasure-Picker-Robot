L'un des points important de l'aspect physique du robot est d'avoir un centre de gravit� assez bas pour �viter � celui-ci de basculer en cas de d�part brusque. Un poids assez important pour coller le robot au sol est aussi important pour ce point. Pour atteindre cet objectif il serait id�al de placer l'ordinateur embarqu� sur l'�tage inf�rieur puisqu'il est le syst�me le plus lourd. Il fut cependant d�cid� de la placer sur l'�tage au-dessus de l'ordinateur afin de lib�rer cet �tage pour placer l'�lectronique et la batterie, r�duisant ainsi la longueur des fils n�cessaires pour se rendre aux moteurs par exemple.
\medbreak
Un autre point important est d'assurer un support solide aux circuits �lectroniques ainsi qu'� la batterie. � cet effet une feuille de contreplaqu� est install�e sur l'�tage inf�rieur afin de visser les circuits sur celle-ci. Bien �videmment des trous pour des vis sont int�gr�s aux circuits et la feuille elle-m�me est fix�e � la base du robot. L'interrupteur principal de l'alimentation est mont� sur le c�t� du robot de fa�on � �tre accessible facilement. Les fusibles sont dispos�s pour pouvoir �tre chang�s facilement en cas de besoin. On minimise ainsi l'espace perdu sur le robot. 