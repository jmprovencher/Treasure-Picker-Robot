La cam�ra Logitech C905 situ�e en hauteur permet de visionner une grande partie de la table. Celle-ci sera branch�e USB � la station de base. Les tests effectu�s jusqu'� pr�sent nous ont permis de contempler la vision limit�e de la table de jeu (figure \ref{f:testCameraMonde}). C'est donc pour cette raison que seule une premi�re approximation sur la position des tr�sors pourra �tre faite. La localisation plus pr�cise de ceux-ci se fera par la cam�ra embarqu�e (idem pour la station de recharge). Voir la section \ref{s:Reperage}.
\medbreak
La d�tection des �les et du robot se fait par la station de base � l'aide de la cam�ra monde. Pour ce faire, le logiciel OpenCV est utilis�. Premi�rement, l'image est d�coup�e de sorte qu'elle ne contienne que la table de jeu. Cet outil tr�s puissant nous permet aussi de d�tecter les contours d'objet � l'int�rieur d'image en utilisant les couleurs pour les distinguer (BGR). Les contours jaunes, bleus, rouges et verts sont d�tect�s afin de rep�rer les �les. Afin d'�viter les erreurs, une relation entre le nombre de coins, le p�rim�tre et l'aire des contours sera v�rifi�e. La validation de cette relation confirmera la pr�sence d'une �le tout en indiquant sa forme et sa couleur (selon le nombre de coins et l'intervalle BGR utilis�). Nous avons d�j� test� ces �tapes et elles semblent plut�t efficaces (figure \ref{f:testDetectionCouleur}). Pour localiser le robot, une forme et une couleur particuli�re seront plac�es au dessus du robot. On proc�dera de fa�on similaire pour le d�tecter.
