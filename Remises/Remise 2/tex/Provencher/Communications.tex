Afin de pouvoir communiquer entre le robot et la station de base, une communication sans-fil est nécessaire. Comme il est important d’établir une connexion avec détection perte de paquet, le protocole TCP a été choisi. Grâce à ce protocole, on s’assure que les paquets perdus seront retransmis afin de ne perdre aucune information dans la communication sans-fil entre le robot et la station de base. Ainsi, en agissant comme serveur, la station de base attendra une connexion client de la part du robot. Une fois cette connexion établie, un échange de fichier JSON contenant des commandes et des paramètres pour le robot s’effectuera. Les avantages d’utiliser un format de fichier JSON pour l’échange est qu’il est simple à implémenter dans plusieurs langages et qu’il est plus facile à parser que le XML.