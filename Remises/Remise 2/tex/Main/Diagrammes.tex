\chapter{Diagrammes}
\label{s:Diagrammes}


\section{Diagramme des fonctionnalit�s}
\label{s:Diagramme_fonctionnalites}

\begin{figure}[htp]
   \centering
   \includegraphics[width=1\textwidth]{pdf/DiagrammeFonctionnalites.pdf}
   \caption{Diagramme des fonctionnalit�s}
   \label{f:diagramme_fonctionnalites}
\end{figure}


\section{Diagramme physique}
\label{s:diagramme_physique}

\begin{figure}[htp]
   \centering
   \includegraphics[width=1\textwidth]{pdf/DiagrammePhysique.pdf}
   \caption{Diagramme physique}
   \label{f:diagramme_physique}
\end{figure}

\section{Diagramme de s�quences}
\label{s:diagramme_sequence}

Les figures \ref{f:diagramme_sequence1} � \ref{f:diagramme_sequence4} sont correspondent aux diagrammes de s�quences.

\begin{figure}[htp]
   \centering
   \includegraphics[width=1\textwidth]{fig/demarageRobot.png}
   \caption{Diagramme de s�quences: d�marage du robot}
   \label{f:diagramme_sequence1}
\end{figure}

\begin{figure}[htp]
   \centering
   \includegraphics[width=1\textwidth]{fig/deplaceTresor.png}
   \caption{Diagramme de s�quences: Deplacement du tr�sor}
   \label{f:diagramme_sequence2}
\end{figure}

\begin{figure}[htp]
   \centering
   \includegraphics[width=1\textwidth]{fig/initCarte.png}
   \caption{Diagramme de s�quences: initialisation de la carte}
   \label{f:diagramme_sequence3}
\end{figure}

\begin{figure}[htp]
   \centering
   \includegraphics[width=1\textwidth]{fig/trouveIleOuTresor.png}
   \caption{Diagramme de s�quences: trouver les �les et tr�sors}
   \label{f:diagramme_sequence4}
\end{figure}
