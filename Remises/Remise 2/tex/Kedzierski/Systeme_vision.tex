Le syst�me de vision est g�r� en grande partie par la station de base, qui re�oit l'information n�c�ssaire par l'entremise de la cam�ra world et celle pr�sente sur le robot. Ce syst�me est programm� en Python et utilise la librairie graphique de traitement d'image OpenCV. 

Le syst�me de vision sert � faire une analyse en profondeur de la table de jeu et � prendre des d�cisions gr�ce aux informations venant du robot. Le syst�me doit donc �tre performant et stable afin de ne pas induire le robot en erreur. 

La cam�ra pr�sente sur le robot ne sera utilis�e que dans deux situations. Premi�rement, elle servira lorsqu'aucun tr�sor est d�tect� par la cam�ra world puisque le robot tentera d'identifier la position des tr�sors gr�ce � son point de vue. Deuxi�mement, la cam�ra du robot servira lors de l'approche d'une cible comme une �le, un tr�sor ou la station de recharge. Donc, une fois le robot d�plac� au point de positionnement devant la cible, la cam�ra embarqu�e enverra des images � la station de base afin de commencer la phase d'approche. Cette phase permet au robot de se positionner de mani�re optimale afin de faciliter l'ex�cution de la t�che demand�e.

Localisation des iles/tresor/robot: 
- parler de la detection des formes avec le nombre de sommet VS detection avec template de forme bcp plus efficace et stable
- Photo HD au debut et pt au debut de chaque phase d'approche
