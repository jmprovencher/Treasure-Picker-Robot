Afin de pouvoir communiquer entre le robot et la station de base, une communication sans-fil est n�cessaire. Comme il est important d'�tablir une connexion avec d�tection perte de paquet, le protocole TCP a �t� choisi. Gr�ce � ce protocole, on s'assure que les paquets perdus seront retransmis afin de ne perdre aucune information dans la communication sans-fil entre le robot et la station de base. L'interface client-serveur est programm�e en Python � l'aide de la librairie socket. Ainsi, en agissant comme serveur, la station de base attendra une connexion client de la part du robot. Une fois cette connexion �tablie, un �change de fichier JSON contenant des commandes et des param�tres pour le robot s'effectuera. Les principaux avantages d'utiliser un format de fichier JSON pour l'�change de commande est qu'il est simple � impl�menter dans plusieurs langages et qu'il est plus facile � parser que le XML. Le fichier JSON contient par exemple une cl�-valeur "commande": "avancer" ainsi que "parametre": 15. Une telle commande permet donc au robot de faire un d�placement avant de 15 centim�tres. Une fois cette t�che compl�t�e, le robot envoie une signal de compl�tion � la station de base et signale du m�me coup qu'il est disponible pour la r�ception d'une nouvelle commande.

Afin de pouvoir recueillir la forme ou la couleur de l'�le vis�e, une requ�te de type GET est effectu�e au serveur des �les. Une fois cette information recueillie de la part du serveur, la forme ou la couleur est envoy� vers l'interface graphique et vers la section de "pathfinding" de la partie logicielle.

