

\section{Diagramme de classe}

Le diagramme de classe � une taille trop importante pour �tre int�gr� au rapport, mais il se retrouve sous forme d'image dans le dossier de remise. Une bref description du flot d'�x�cution de la routine suivra afin de mieu comprendre la fonction de chaque classe. \medbreak
Du c�t� de la station de base, le main initialise l'interface. Les boutons permettent par la suite de commencer une routine en initialisant la classe StationBase avec la table de jeux et la routine voulu. Ce thread est le controleur de la station. Il commence par initialiser les autres threads: 
\begin{enumerate}
\item StationServeur. Ce thread initialise une connection TCP avec le robot (avec TCPServeur). Il est en charge d'envoyer et de traiter tout la communication entre la station de base et le robot. Ceci est effectu� par l'envoie de fichier JSON qui sont cr�� par la classe RequeteJSON.
\item FeedVideoStation. Ce thread a comme simple tache de lire les images de la cam�ra world.
\item AnalyseImageWorld. Ce thread ce sers des classes de Detection pour rep�rer les �les, tr�sors et le robot. Ceux-ci utilise les diff�rentes intervalle de couleurs correspondante � chaque table dans IntervalleCouleur. Une fois trouv�e, Les �l�ments cartographiques sont ajout� � la classe Carte.
\item ImageVirtuelle. Ce thread est en charge de dessiner sur l'image obtenue afin de l'afficher dans l'interface avec les informations voulu (trajectoire pr�vue, position robot, etc.). La classe repr�sente donc tout simplement la carte virtuelle.
\end{enumerate}
Suite � l'initialisation de StationBase, Interface initialise un dernier thread (AfficherImageVirtuelle). Celui-ci a comme seul t�che de mettre � jour le feed vid�o (ImageVirtuelle) dans l'interface.  \medbreak
La classe StationBase d�s le d�but de l'�x�cution initialise la classe Trajectoire avec GrilleCellule. Celle-ci repr�sente la matrice de position que peut prendre le robot. Lors de la demande d'un trajet, Trajectoire appelle AlgorithmeTrajectoire avec la grille de cellule. Cette classe trouvera la trajectoire demand�. \medbreak
Finalement, la classe RedirigeurTexte sers simplement � rediriger les print dans la boite de texte de l'interface.\bigbreak
Du cot� du robot, le controleur est la classe Robot. La communication se d�roule comme avec la station de base avec les classes (RequeteJSON, RobotClient, TCPClient). La classe UARTDriver est utilis� pour envoyer des commandes par UART. Tandis que la classe LectureUART est utilis� pour lire et traiter les donn�es envoy� par le UART. \medbreak
Lorsque le robot d�bute les phase d'alignement, il se sert de AnalyseImageEmbarquee et la classe de detection correspondante afin de trouver les d�placements requis. \medbreak
La classe, RobotService est utilis�e pour obtenir l'indice � l'aide de la lettre manchester.\bigbreak
Finalement, la classe ConfigPath est utilis�e afin de transformer les path locale en path absolu. 