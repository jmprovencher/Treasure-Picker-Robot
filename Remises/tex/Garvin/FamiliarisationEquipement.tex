La cam�ra Logitech C905 situ� en hauteur permet de visionner une grande partie de la table. Celle-ci sera branch� USB � la station de base. Les tests effectu� jusqu'� pr�sent nous on permis de contempler la vision limiter de la table de jeux (figure \ref{f:testCameraMonde}). C'est donc pour cette raison que seul une premi�re approximation sur la position des tr�sors pourra �tre faite. La localisation plus pr�cise de ceux-ci se fera par la cam�ra embarqu� (idem pour la station de recharge). Voir la section \ref{s:Reperage}.
\medbreak
La d�tection des �les et du robot se fait par la station de base � l'aide de la cam�ra monde. Pour ce faire, le logiciel OpenCV est utilis�. Premi�rement, l'image est d�coup� de sorte � ce qu'elle ne contienne que la table de jeux. Cet outil tr�s puissant nous permet aussi de d�tecter les contours d'objet � l'int�rieur d'image en utilisant les couleur pour les distinguer (BGR). Les contours jaune, bleu, rouge et vert sont d�tect� afin de rep�rer les �les. Afin d'�viter les erreurs, une relation entre le nombre de coin, le p�rim�tre et l'air des contours sera v�rifi�. La validation de cette relation confirmera la pr�sence d'une �le tout en indiquant sa forme et sa couleur (selon le nombre de coin et l�intervalle BGR utilis�). Nous avons d�j� test� ces �tapes et elle semble plut�t efficace (figure \ref{f:testDetectionCouleur}). Pour localiser le robot, une forme et couleur particuli�re sera plac� au dessus du robot. On proc�dera de fa�on similaire pour le d�tecter.
