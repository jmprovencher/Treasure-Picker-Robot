Afin de commander les moteurs du robot, il sera n�cessaire d'avoir une routine d'asservissement. Afin de communiquer les commandes avec les servomoteurs, l'Arduino Mega est choisi. Celui-ci poss�de une vitesse d'horloge de 16 MHz et 256kB de m�moire flash pour 44\$ \cite{MEG00}, ce qui est un bon �quilibre performance/prix pour les besoins du projet.

Les moteurs utilis�s pour tourner les roues du robot poss�dent 6400 valeurs de position par rotation. Si une interruption �tait effectu�e � chaque modification de cette valeur, on risque d'emp�cher l'ex�cution compl�te d'une boucle d'asservissement entre deux interruptions. Donc, si une interruption est lev�e � chaque modification de la position, il faudra s'assurer que le calcul de la vitesse de rotation se fasse apr�s un nombre multiple d'interruptions, avec un compteur de temps.

Pour ce qui a trait � la communication avec les servomoteurs, un shield Adafruit se connectant directement sur l'Arduino Mega est utilis�. Avec celui-ci, il est possible de g�n�rer des ondes modul�es (PWM) servant � contr�ller les quatres roues individuellement, avec chacun une r�solution de 8 bits et jusqu'� 1.2 amp�res d'alimentation \cite{SLD00}. Cette division de la commande est utile lors d'un d�placement en diagonale, et pourrait �galement �tre utile pour continuer le fonctionnement dans le cas d'une roue d�fectueuse.

En r�sum�, l'ordinateur envoie des instructions de direction au microcontr�lleur Arduino, qui execute alors une routine de d�placement. Celle-ci est bas�e sur un asservissement de vitesse, d�termin�e par des interruptions lors de la rotation des servomoteurs. Les commandes de l'Arduino sont alors communiqu�es aux roues � l'aide d'un shield Adafruit.