Afin de commander les moteurs du robot, il est n�cessaire d'avoir une routine d'asservissement. Pour envoyer les commandes aux moteurs DC, l'Arduino Mega est choisi. Celui-ci poss�de une vitesse d'horloge de 16 MHz et 256kB de m�moire flash pour 45\$ \cite{MEG00}, ce qui est un bon �quilibre performance/prix pour les besoins du projet.
\medbreak
Les moteurs utilis�s pour entra�ner les roues du robot poss�dent 6400 valeurs de position par rotation. Si une interruption est effectu�e � chaque modification de cette valeur, on risque d'emp�cher l'ex�cution compl�te d'une boucle d'asservissement entre deux interruptions. Donc, si une interruption est lev�e � chaque modification de la position, il faut s'assurer que le calcul de la vitesse de rotation se fasse apr�s un nombre multiple d'interruptions, avec un compteur de temps.
\medbreak
Pour ce qui a trait � la communication avec les moteurs DC, un shield Adafruit se connectant directement sur l'Arduino Mega est utilis�. Avec celui-ci, il est possible de g�n�rer des ondes modul�es (PWM) servant � contr�ler les quatre roues individuellement, avec chacune une r�solution de 8 bits et jusqu'� 1.2A par canal\cite{SLD00}. Cette division de la commande est utile lors d'un d�placement en diagonale, et pourrait �galement �tre utile pour continuer le fonctionnement dans le cas d'une roue d�fectueuse. Ce dispositif est utilis� pour contr�ler les moteurs pour les premiers tests, bien �videmment l'utilisation du pont en H fournit fera aussi l'objet de tests lorsque la batterie et l'alimentation seront disponibles.
\medbreak
En r�sum�, l'ordinateur envoie des instructions de direction au microcontr�leur Arduino, qui ex�cute alors une routine de d�placement. Celle-ci est bas�e sur un asservissement de vitesse, d�termin�e par des interruptions lors de la rotation des servomoteurs. Les commandes de l'Arduino sont alors communiqu�es aux roues � l'aide d'un shield Adafruit permettant de leur d�livrer de la puissance.