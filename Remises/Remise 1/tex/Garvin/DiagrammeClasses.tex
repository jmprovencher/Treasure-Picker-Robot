
La figure \ref{f:Diagramme_Classes} repr�sente le diagramme de classes suite � la premi�re it�ration. La structure est susceptible de changer suite aux prochaines it�rations, mais voici une br�ve description de l'architecture actuelle de l'aspect logiciel du projet. \par
\medbreak
La section de gauche du diagramme sera impl�ment�e sur la station de base, tandis que la section de droite sera impl�ment�e sur le robot. Ces deux syst�me pourront communiquer entre eux � l'aide des classes CommunicationRobot et CommunicationStationBase. \par
\medbreak
En ce qui concerne la station de base, le contr�leur du syst�me est repr�sent� par la classe StationBase. La classe AnalyseImageWorld analyse les images re�ues de la CameraMonde et g�n�re une carte sch�matique de la table (Carte) � l'aide d'imagerie. La carte est compos�e de divers �l�ments qui h�ritent tous de la classe Position. Les trajectoires du robot seront calcul�es dans la classe TrajetStation � l'aide des informations de la classe Carte.  \par
\medbreak
Pour ce qui est du robot, il est aussi compos� d'un contr�leur (Robot). Les demandes de mouvement que devra effectuer le robot passeront toutes par la classe Action, qui les acheminera au microcontr�leur Arduino Mega 2560. Lorsque le robot sera pr�s de la destination, TrajetRobot calculera les trajets (� l'aide de Destination et d'imagerie effectu�e dans AnalyseImageEmbarquer). 


    
   
