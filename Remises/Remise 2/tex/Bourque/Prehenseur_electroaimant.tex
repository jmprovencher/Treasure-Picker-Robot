Le pr�henseur est constitu� d'un �lectroaimant situ� sur un bras m�canique. Cette �lectroaimant est aliment� par le condensateur de 5F. Lorsqu'un courant circule dans l'�lectroaimant, il produit une force d'attraction des tr�sors. Afin de conserver l��nergie dans le condensateur le plus longtemps possible, le courant de d�charge dans l��lectroaimant doit �tre limit�. En effet, sans cela, le condensateur se d�chargera extr�mement rapidement dans l'aimant, car sa r�sistance interne est de 12 ohms. 
\medbreak
Afin de limiter ce courant, nous utilisons un MOSFET dans lequel nous envoyons une onde carr� de largeur d�impulsion variable dans sa base et permet de contr�ler ce courant. Un fort courant doit d'abord �tre envoy� dans l'aimant afin d'attirer le tr�sor. Ensuite, le circuit magn�tique est ferm� et un courant plus faible peut donc �tre envoy�. Cette fa�on de proc�der est donc utilis�e afin de conserver l'�nergie du condensateur plus longtemps.
\medbreak
�galement, le syst�me comprend un interrupteur �lectronique afin de permettre de couper la d�charge de courant dans l��lectroaimant lorsque le d�placement vers l'�le cible est termin�. En ouvrant l�interrupteur, il n'y a plus de courant dans l'�lectroaimant et le force qui permet de tenir le tr�sor disparait. Le tr�sor tombe ainsi sur l'�le.
